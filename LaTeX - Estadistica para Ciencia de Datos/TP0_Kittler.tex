%% LyX 2.3.7 created this file.  For more info, see http://www.lyx.org/.
%% Do not edit unless you really know what you are doing.
\documentclass[spanish]{article}
\usepackage[T1]{fontenc}
\usepackage{amstext}

\makeatletter
%%%%%%%%%%%%%%%%%%%%%%%%%%%%%% User specified LaTeX commands.
\usepackage{palatino}

\makeatother

\usepackage{babel}
\addto\shorthandsspanish{\spanishdeactivate{~<>}}

\begin{document}
\title{Trabajo Pr\'{a}ctico 0: El algoritmo de umbralizaci\'{o}n de Kittler}
\author{Ph. D. Sa\'{u}l Calder\'{o}n Ram\'{\i}rez \\
Instituto Tecnol\'{o}gico de Costa Rica, \\
Escuela de Computaci\'{o}n\\
PAttern Recongition and MAchine Learning Group (PARMA-Group)}

\maketitle
\textbf{Fecha de entrega: }Domingo 21 de Abril.

\textbf{Entrega}: Un archivo .zip con el c\'{o}digo fuente LaTeX o
Lyx, el pdf, y un script en jupyter, debidamente documentado. El script
en jupyter debe estar escrito usando pytorch. A trav\'{e}s del TEC-digital.

\textbf{Modo de trabajo}: Grupos de 3 personas.

En el presente trabajo pr\'{a}ctico se introducir\'{a} el problema
de la clasificaci\'{o}n a trav\'{e}s del estudio del concepto de m\'{a}xima
verosimilitud. Se realizar\'{a} un repaso de la teor\'{\i}a b\'{a}sica
relacionada con los fen\'{o}menos aleatorios con distribuci\'{o}n
Gaussiana, para facilitar el an\'{a}lisis de la funci\'{o}n de verosimilitud.
Posteriormente se visitar\'{a} el problema de la segmentaci\'{o}n
de im\'{a}genes desde un enfoque de m\'{a}xima verosimilitud, donde
se desarrollar\'{a} el algoritmo de Kittler \cite{kittler1986minimum}.
El estudiante implementar\'{a} tal algoritmo y analizar\'{a} los resultados
respecto a los planteos te\'{o}ricos introducidos previamente. 

\section{Implementaci\'{o}n del algoritmo de Kittler}
\begin{enumerate}
\item Implemente el algoritmo de Kittler, y realice una prueba con la imagen
de entrada provista, aplicando posteriormente el umbral \'{o}ptimo
obtenido. 

\begin{enumerate}
\item \textbf{(20 puntos)} Implemente una funcion \emph{calcular\_momentos\_estadisticos(T,
p)} la cual reciba un umbral candidato $T$ y una funcion de densidad
$p$, y retorne todos los parametros de la funci\'{o}n. Comente su
implementacion con detalle en este informe. 
\begin{enumerate}
\item Dise\~{n}e al menos 2 pruebas unitarias donde verifique el funcionamiento
correcto. Detalle en este documento el dise\~{n}o de tales pruebas
y los resultados, indicando si son los esperados. 
\end{enumerate}
\item \textbf{(20 puntos)} Implemente la funci\'{o}n \emph{calcular\_costo\_J(T)
}la cual calcule el costo del umbral candidato $T$. Comente su implementacion
con detalle en este informe. 
\begin{enumerate}
\item Dise\~{n}e al menos 2 pruebas unitarias donde verifique el funcionamiento
correcto. Detalle en este documento el dise\~{n}o de tales pruebas
y los resultados, indicando si son los esperados.
\end{enumerate}
\item \textbf{(20 puntos)} Basado en ambas funciones, implemente la funci\'{o}n
\emph{calcular\_T\_optimo\_Kittler(Imagen) }la cual retorne el $T$
optimo para umbralizar la imagen recibida, adem\'{a}s de la imagen
umbralizada. 
\item Aplique el algoritmo de Kittler en la imagen \emph{cuadro1\_005.bmp,
}provista.

\begin{enumerate}
\item Grafique el histograma normalizado de la imagen de entrada provista.
\item Grafique la funci\'{o}n $J(T)$, y documente el valor $T=\tau$ que
logra el valor m\'{\i}nimo de $J(T)$, junto con las medias y varianzas
de las dos funciones Gaussianas superpuestas. Son coherentes tales
valores con el histograma graficado en el punto anterior?

\begin{enumerate}
\item El valor \'{o}ptimo en el caso de esta imagen debe ser cercano a $\tau=168$,
con $\mu_{1}=149.45$, $\mu_{2}=219.49$ $\sigma_{1}^{2}=15.36$ y
$\sigma_{2}^{2}=10.05$.
\end{enumerate}
\item Lograr\'{\i}a el umbral \'{o}ptimo $\tau$ obtenido umbralizar satisfactoriamente
la imagen de prueba? Umbralice la imagen de entrada provista y muestre
los resultados.
\begin{enumerate}
\item Asigne con valor de 255 los pixeles del cuadrado (clase \emph{foreground}),
y 0 los del fondo (clase \emph{background}). 
\end{enumerate}
\end{enumerate}
\item (\textbf{25 puntos}) Pruebe la implementacion del algortimo de Kittler
para detectar la actividad de voz humana en un audio, usando el audio
de prueba provisto \emph{contaminated\_audio.wav.}
\begin{enumerate}
\item Grafique el histograma del audio leido, y argumente si es apropiado
usar el algoritmo de Kittler o no.
\item Grafique la funci\'{o}n $J(T)$, y documente el valor $T=\tau$ que
logra el valor m\'{\i}nimo de $J(T)$, junto con las medias y varianzas
de las dos funciones Gaussianas superpuestas. Son coherentes tales
valores con el histograma graficado en el punto anterior?
\item Lograr\'{\i}a el umbral \'{o}ptimo $\tau$ obtenido partir satisfactoriamente
el sonido de prueba en 3 segmentos (2 de actividades de audio y 1
segmento sin actividad )? Umbralice el audio provisto y muestre los
resultados.
\end{enumerate}
\end{enumerate}
\item \textbf{(15 puntos)} La distancia de Bhattacharyya compara dos funciones
de densidad de probabilidad $p\left(x\right)$ y $q\left(x\right)$:
\[
D_{\textrm{JS}}\left(p,q\right)=-\ln\left(\sum_{x\in X}\sqrt{p\left(x\right)q\left(x\right)}\right)
\]

\begin{enumerate}
\item Implemente la funci\'{o}n \emph{calcular\_bhattacharyya\_distance(p,q),
}para comparar\emph{ }las funciones de densidad estimada con el ajuste
del modelo mixto Gaussiano con Kittler $p\left(x\right)$ y la aproximacion
de la densidad con el histograma de los datos $q\left(x\right)$ para
las dos pruebas realizadas con la imagen y el audio. 
\item Explique la relaci\'{o}n entre la distancia de Bhattacharyya y el
proceso de estimaci\'{o}n de los par\'{a}metros \'{o}ptimos implementado
en el algoritmo de Kittler \textquestiondown Que sucede cuando la
distancia de Bhattacharyya entre el histograma de los datos y el modelo
estimado crece o decrece ?
\end{enumerate}
\item \textbf{(10 puntos extra)} Calcule el umbral \'{o}ptimo con el algoritmo
de Kittler, y umbralice la imagen de \emph{trackedCell15.tif} provista
documentando los resultados. Muestre la imagen umbralizada y el histograma
de la misma. 
\begin{enumerate}
\item Usando la imagen, su histograma, y la matriz de confusion para la
clase \emph{foreground,} explique el porqu\'{e} del resultado obtenido.
\item Como modificar\'{\i}a el algoritmo de Kittler para mejorar el resultado
de la umbralizaci\'{o}n? Puede usar recursos bibliogr\'{a}ficos externos.
\end{enumerate}
\item \textbf{(10 puntos extra)} Tanto para el audio y la imagen, implementa y pruebe la optimizaci\'{o}n con un paquete como \textit{Optuna} o \textit{Weights and Biases}. Reporte los resultados y com\'{e}ntelos.
\end{enumerate}
\bibliographystyle{plain}
\bibliography{BIB}

\end{document}
